\documentclass[]{article}
\usepackage{lmodern}
\usepackage{amssymb,amsmath}
\usepackage{ifxetex,ifluatex}
\usepackage{fixltx2e} % provides \textsubscript
\ifnum 0\ifxetex 1\fi\ifluatex 1\fi=0 % if pdftex
  \usepackage[T1]{fontenc}
  \usepackage[utf8]{inputenc}
\else % if luatex or xelatex
  \ifxetex
    \usepackage{mathspec}
  \else
    \usepackage{fontspec}
  \fi
  \defaultfontfeatures{Ligatures=TeX,Scale=MatchLowercase}
\fi
% use upquote if available, for straight quotes in verbatim environments
\IfFileExists{upquote.sty}{\usepackage{upquote}}{}
% use microtype if available
\IfFileExists{microtype.sty}{%
\usepackage{microtype}
\UseMicrotypeSet[protrusion]{basicmath} % disable protrusion for tt fonts
}{}
\usepackage[margin=1in]{geometry}
\usepackage{hyperref}
\hypersetup{unicode=true,
            pdftitle={Deconfounder with single outcome},
            pdfborder={0 0 0},
            breaklinks=true}
\urlstyle{same}  % don't use monospace font for urls
\usepackage{color}
\usepackage{fancyvrb}
\newcommand{\VerbBar}{|}
\newcommand{\VERB}{\Verb[commandchars=\\\{\}]}
\DefineVerbatimEnvironment{Highlighting}{Verbatim}{commandchars=\\\{\}}
% Add ',fontsize=\small' for more characters per line
\usepackage{framed}
\definecolor{shadecolor}{RGB}{248,248,248}
\newenvironment{Shaded}{\begin{snugshade}}{\end{snugshade}}
\newcommand{\AlertTok}[1]{\textcolor[rgb]{0.94,0.16,0.16}{#1}}
\newcommand{\AnnotationTok}[1]{\textcolor[rgb]{0.56,0.35,0.01}{\textbf{\textit{#1}}}}
\newcommand{\AttributeTok}[1]{\textcolor[rgb]{0.77,0.63,0.00}{#1}}
\newcommand{\BaseNTok}[1]{\textcolor[rgb]{0.00,0.00,0.81}{#1}}
\newcommand{\BuiltInTok}[1]{#1}
\newcommand{\CharTok}[1]{\textcolor[rgb]{0.31,0.60,0.02}{#1}}
\newcommand{\CommentTok}[1]{\textcolor[rgb]{0.56,0.35,0.01}{\textit{#1}}}
\newcommand{\CommentVarTok}[1]{\textcolor[rgb]{0.56,0.35,0.01}{\textbf{\textit{#1}}}}
\newcommand{\ConstantTok}[1]{\textcolor[rgb]{0.00,0.00,0.00}{#1}}
\newcommand{\ControlFlowTok}[1]{\textcolor[rgb]{0.13,0.29,0.53}{\textbf{#1}}}
\newcommand{\DataTypeTok}[1]{\textcolor[rgb]{0.13,0.29,0.53}{#1}}
\newcommand{\DecValTok}[1]{\textcolor[rgb]{0.00,0.00,0.81}{#1}}
\newcommand{\DocumentationTok}[1]{\textcolor[rgb]{0.56,0.35,0.01}{\textbf{\textit{#1}}}}
\newcommand{\ErrorTok}[1]{\textcolor[rgb]{0.64,0.00,0.00}{\textbf{#1}}}
\newcommand{\ExtensionTok}[1]{#1}
\newcommand{\FloatTok}[1]{\textcolor[rgb]{0.00,0.00,0.81}{#1}}
\newcommand{\FunctionTok}[1]{\textcolor[rgb]{0.00,0.00,0.00}{#1}}
\newcommand{\ImportTok}[1]{#1}
\newcommand{\InformationTok}[1]{\textcolor[rgb]{0.56,0.35,0.01}{\textbf{\textit{#1}}}}
\newcommand{\KeywordTok}[1]{\textcolor[rgb]{0.13,0.29,0.53}{\textbf{#1}}}
\newcommand{\NormalTok}[1]{#1}
\newcommand{\OperatorTok}[1]{\textcolor[rgb]{0.81,0.36,0.00}{\textbf{#1}}}
\newcommand{\OtherTok}[1]{\textcolor[rgb]{0.56,0.35,0.01}{#1}}
\newcommand{\PreprocessorTok}[1]{\textcolor[rgb]{0.56,0.35,0.01}{\textit{#1}}}
\newcommand{\RegionMarkerTok}[1]{#1}
\newcommand{\SpecialCharTok}[1]{\textcolor[rgb]{0.00,0.00,0.00}{#1}}
\newcommand{\SpecialStringTok}[1]{\textcolor[rgb]{0.31,0.60,0.02}{#1}}
\newcommand{\StringTok}[1]{\textcolor[rgb]{0.31,0.60,0.02}{#1}}
\newcommand{\VariableTok}[1]{\textcolor[rgb]{0.00,0.00,0.00}{#1}}
\newcommand{\VerbatimStringTok}[1]{\textcolor[rgb]{0.31,0.60,0.02}{#1}}
\newcommand{\WarningTok}[1]{\textcolor[rgb]{0.56,0.35,0.01}{\textbf{\textit{#1}}}}
\usepackage{graphicx}
% grffile has become a legacy package: https://ctan.org/pkg/grffile
\IfFileExists{grffile.sty}{%
\usepackage{grffile}
}{}
\makeatletter
\def\maxwidth{\ifdim\Gin@nat@width>\linewidth\linewidth\else\Gin@nat@width\fi}
\def\maxheight{\ifdim\Gin@nat@height>\textheight\textheight\else\Gin@nat@height\fi}
\makeatother
% Scale images if necessary, so that they will not overflow the page
% margins by default, and it is still possible to overwrite the defaults
% using explicit options in \includegraphics[width, height, ...]{}
\setkeys{Gin}{width=\maxwidth,height=\maxheight,keepaspectratio}
\IfFileExists{parskip.sty}{%
\usepackage{parskip}
}{% else
\setlength{\parindent}{0pt}
\setlength{\parskip}{6pt plus 2pt minus 1pt}
}
\setlength{\emergencystretch}{3em}  % prevent overfull lines
\providecommand{\tightlist}{%
  \setlength{\itemsep}{0pt}\setlength{\parskip}{0pt}}
\setcounter{secnumdepth}{5}
% Redefines (sub)paragraphs to behave more like sections
\ifx\paragraph\undefined\else
\let\oldparagraph\paragraph
\renewcommand{\paragraph}[1]{\oldparagraph{#1}\mbox{}}
\fi
\ifx\subparagraph\undefined\else
\let\oldsubparagraph\subparagraph
\renewcommand{\subparagraph}[1]{\oldsubparagraph{#1}\mbox{}}
\fi

%%% Use protect on footnotes to avoid problems with footnotes in titles
\let\rmarkdownfootnote\footnote%
\def\footnote{\protect\rmarkdownfootnote}

%%% Change title format to be more compact
\usepackage{titling}

% Create subtitle command for use in maketitle
\providecommand{\subtitle}[1]{
  \posttitle{
    \begin{center}\large#1\end{center}
    }
}

\setlength{\droptitle}{-2em}

  \title{Deconfounder with single outcome}
    \pretitle{\vspace{\droptitle}\centering\huge}
  \posttitle{\par}
    \author{}
    \preauthor{}\postauthor{}
    \date{}
    \predate{}\postdate{}
  

\begin{document}
\maketitle

{
\setcounter{tocdepth}{2}
\tableofcontents
}
To begin, install the following libraries from OHDSI github. Packages
only need to be installed once.

\begin{Shaded}
\begin{Highlighting}[]
\NormalTok{devtools}\OperatorTok{::}\KeywordTok{install\_github}\NormalTok{(}\StringTok{"ohdsi/SqlRender"}\NormalTok{)}
\NormalTok{devtools}\OperatorTok{::}\KeywordTok{install\_github}\NormalTok{(}\StringTok{"ohdsi/DatabaseConnector"}\NormalTok{)}
\NormalTok{devtools}\OperatorTok{::}\KeywordTok{install\_github}\NormalTok{(}\StringTok{"ohdsi/FeatureExtraction"}\NormalTok{)}
\NormalTok{devtools}\OperatorTok{::}\KeywordTok{install\_github}\NormalTok{(}\StringTok{"ohdsi/PatientLevelPrediction"}\NormalTok{)}
\end{Highlighting}
\end{Shaded}

Connect to your database using the \emph{DatabaseConnector} package. For
details about the \emph{createConnectionDetails} function, run
?createConnectionDetails or help(createConnectionDetails) in the R
console.

\begin{Shaded}
\begin{Highlighting}[]
\NormalTok{connectionDetails =}\StringTok{ }\NormalTok{DatabaseConnector}\OperatorTok{::}\KeywordTok{createConnectionDetails}\NormalTok{(}\DataTypeTok{dbms =} \StringTok{"sql server"}\NormalTok{,}
                                             \DataTypeTok{server =} \StringTok{"omop.dbmi.columbia.edu"}\NormalTok{)}
\NormalTok{connection =}\StringTok{ }\NormalTok{DatabaseConnector}\OperatorTok{::}\KeywordTok{connect}\NormalTok{(connectionDetails)}
\end{Highlighting}
\end{Shaded}

Specify the following database schemas. The \emph{targetCohortTable} is
the name of the cohort table. Change the \emph{targetCohortId} when
create a new cohort.The \emph{drugExposureTable} is the name of the
table where drug exposure of the cohort will be stored. The
\emph{measurementTable} is the name of the table where both pre-exposure
and post-exposure measurements of the cohort will be stored.

\begin{Shaded}
\begin{Highlighting}[]
\NormalTok{cdmDatabaseSchema =}\StringTok{ "ohdsi\_cumc\_deid\_2020q2r2.dbo"}
\NormalTok{cohortDatabaseSchema =}\StringTok{ "ohdsi\_cumc\_deid\_2020q2r2.results"}
\NormalTok{targetCohortTable =}\StringTok{ "DECONFOUNDER\_COHORT"}
\NormalTok{drugExposureTable =}\StringTok{ "SAMPLE\_COHORT\_DRUG\_EXPOSURE"}
\NormalTok{measurementTable =}\StringTok{ "SAMPLE\_COHORT\_MEASUREMENT"}
\NormalTok{targetCohortId =}\StringTok{ }\DecValTok{1}
\end{Highlighting}
\end{Shaded}

The \emph{conditionConceptIds} is the conditions of interest. The date
of diagnosis of the condition is the cohort start date. The
\emph{measurementConceptId} is the outcome of interest. Currently
outcome only supports lab measurement (continuous). For example, if the
study is to estimate the treatment effect of drugs taken by a potassium
disorder (both hypo- and hyperkalemia) cohort,

\begin{Shaded}
\begin{Highlighting}[]
\NormalTok{conditionConceptIds \textless{}{-}}\StringTok{ }\KeywordTok{c}\NormalTok{(}\DecValTok{434610}\NormalTok{,}\DecValTok{437833}\NormalTok{) }\CommentTok{\# Hypo and hyperkalemia}
\NormalTok{measurementConceptId \textless{}{-}}\StringTok{ }\KeywordTok{c}\NormalTok{(}\DecValTok{3023103}\NormalTok{) }\CommentTok{\# serum potassium}
\end{Highlighting}
\end{Shaded}

The \emph{observationWindowBefore} \emph{observationWindowAfter} are the
time window (in days) to query for pre-treatment measurement and
post-treatment measurement respectively. The \emph{drugWindow} is the
post-treatment time window (in days) to query for drug exposures from
the DRUG\_EXPOSURE table, and value 0 means only drugs prescribed on the
same day as the day of diagnosis will be included. If
drugWindow\textgreater0, then drugs prescribed post diagnosis will also
be included.

\begin{Shaded}
\begin{Highlighting}[]
\NormalTok{observationWindowBefore \textless{}{-}}\StringTok{ }\DecValTok{7}
\NormalTok{observationWindowAfter \textless{}{-}}\StringTok{ }\DecValTok{30}
\NormalTok{drugWindow \textless{}{-}}\StringTok{ }\DecValTok{0}
\end{Highlighting}
\end{Shaded}

To create the cohort and extract drug exposures and pre-treatment and
post-treatment lab values. The output of \emph{generateData} are two
tables \emph{measFilename} and \emph{drugFilename} stored at
\emph{dataFolder}.

\begin{Shaded}
\begin{Highlighting}[]
\NormalTok{measFilename \textless{}{-}}\StringTok{ "meas.csv"}
\NormalTok{drugFilename \textless{}{-}}\StringTok{ "drug.csv"}
\NormalTok{dataFolder \textless{}{-}}\StringTok{ "path/to/datafolder"}
\NormalTok{Deconfounder}\OperatorTok{::}\KeywordTok{generateData}\NormalTok{(connection,}
\NormalTok{             cdmDatabaseSchema,}
             \DataTypeTok{oracleTempSchema =} \OtherTok{NULL}\NormalTok{,}
             \DataTypeTok{vocabularyDatabaseSchema =}\NormalTok{ cdmDatabaseSchema,}
\NormalTok{             cohortDatabaseSchema,}
\NormalTok{             targetCohortTable,}
\NormalTok{             drugExposureTable,}
\NormalTok{             measurementTable,}
\NormalTok{             conditionConceptIds,}
\NormalTok{             measurementConceptId,}
\NormalTok{             observationWindowBefore,}
\NormalTok{             observationWindowAfter,}
\NormalTok{             drugWindow,}
             \DataTypeTok{createTargetCohortTable =}\NormalTok{ T,}
             \DataTypeTok{createTargetCohort =}\NormalTok{ T,}
             \DataTypeTok{extractFeature =}\NormalTok{ T,}
             \DataTypeTok{targetCohortId=}\NormalTok{targetCohortId,}
\NormalTok{             dataFolder,}
\NormalTok{             drugFilename,}
\NormalTok{             measFilename)}
\end{Highlighting}
\end{Shaded}

The rest of the algorithm is implemented with python. First, specify the
python to use using the \emph{reticulate} package. For more

\begin{Shaded}
\begin{Highlighting}[]
\NormalTok{reticulate}\OperatorTok{::}\KeywordTok{use\_condaenv}\NormalTok{(}\StringTok{"deconfounder\_py3"}\NormalTok{, }\DataTypeTok{required =} \OtherTok{TRUE}\NormalTok{)}
\end{Highlighting}
\end{Shaded}

First, preprocess the data for the deconfounder.

\begin{Shaded}
\begin{Highlighting}[]
\NormalTok{Deconfounder}\OperatorTok{::}\KeywordTok{preprocessingData}\NormalTok{(dataFolder, measFilename, drugFilename, drugWindow)}
\end{Highlighting}
\end{Shaded}

Specify the factor model to use, currently supporting ``PMF'' (poisson
matrix factoriation) and ``DEF'' (two-layer deep exponential family).

\begin{Shaded}
\begin{Highlighting}[]
\NormalTok{factorModel \textless{}{-}}\StringTok{ \textquotesingle{}DEF\textquotesingle{}}
\NormalTok{outputFolder \textless{}{-}}\StringTok{ "path/to/outputFolder"}
\end{Highlighting}
\end{Shaded}

Next, fit the deconfounder to estimate average treatment effect (ATE).

\begin{Shaded}
\begin{Highlighting}[]
\NormalTok{Deconfounder}\OperatorTok{::}\KeywordTok{fitDeconfounder}\NormalTok{(}\DataTypeTok{data\_dir=}\NormalTok{dataFolder,}
                \DataTypeTok{save\_dir=}\NormalTok{outputFolder,}
                \DataTypeTok{factor\_model=}\NormalTok{factorModel,}
                \DataTypeTok{learning\_rate=}\FloatTok{0.0001}\NormalTok{,}
                \DataTypeTok{max\_steps=}\KeywordTok{as.integer}\NormalTok{(}\DecValTok{100000}\NormalTok{),}
                \DataTypeTok{latent\_dim=}\KeywordTok{as.integer}\NormalTok{(}\DecValTok{1}\NormalTok{), }
                \DataTypeTok{layer\_dim=}\KeywordTok{c}\NormalTok{(}\KeywordTok{as.integer}\NormalTok{(}\DecValTok{20}\NormalTok{), }\KeywordTok{as.integer}\NormalTok{(}\DecValTok{4}\NormalTok{)), }
                \DataTypeTok{batch\_size=}\KeywordTok{as.integer}\NormalTok{(}\DecValTok{1024}\NormalTok{),}
                \DataTypeTok{num\_samples=}\KeywordTok{as.integer}\NormalTok{(}\DecValTok{1}\NormalTok{), }
                \DataTypeTok{holdout\_portion=}\FloatTok{0.5}\NormalTok{, }
                \DataTypeTok{print\_steps=}\KeywordTok{as.integer}\NormalTok{(}\DecValTok{50}\NormalTok{),}
                \DataTypeTok{tolerance=}\KeywordTok{as.integer}\NormalTok{(}\DecValTok{3}\NormalTok{), }
                \DataTypeTok{num\_confounder\_samples=}\KeywordTok{as.integer}\NormalTok{(}\DecValTok{30}\NormalTok{), }
                \DataTypeTok{CV=}\KeywordTok{as.integer}\NormalTok{(}\DecValTok{5}\NormalTok{), }
                \DataTypeTok{outcome\_type=}\StringTok{\textquotesingle{}linear\textquotesingle{}}
\NormalTok{)}
\end{Highlighting}
\end{Shaded}

To visualize the estimated ATE, plot the mean and 95 CI as follows:

\begin{Shaded}
\begin{Highlighting}[]
\KeywordTok{library}\NormalTok{(ggplot2)}
\NormalTok{resFolder \textless{}{-}}\StringTok{ "path/to/resultsFolder"}
\NormalTok{stats \textless{}{-}}\StringTok{ }\KeywordTok{read.csv}\NormalTok{(}\DataTypeTok{file =} \KeywordTok{file.path}\NormalTok{(resFolder, }\StringTok{"treatment\_effects\_stats.csv"}\NormalTok{))}

\NormalTok{stats}\OperatorTok{$}\NormalTok{drug\_name \textless{}{-}}\StringTok{ }\KeywordTok{factor}\NormalTok{(stats}\OperatorTok{$}\NormalTok{drug\_name, }\DataTypeTok{levels =}\NormalTok{ stats}\OperatorTok{$}\NormalTok{drug\_name[}\KeywordTok{order}\NormalTok{(}\OperatorTok{{-}}\NormalTok{stats}\OperatorTok{$}\NormalTok{mean)])}
\NormalTok{p2 \textless{}{-}}\StringTok{ }\KeywordTok{ggplot}\NormalTok{(stats, }\KeywordTok{aes}\NormalTok{(drug\_name, mean)) }\OperatorTok{+}\StringTok{ }\KeywordTok{theme\_gray}\NormalTok{(}\DataTypeTok{base\_size=}\DecValTok{10}\NormalTok{)}
\NormalTok{p2 }\OperatorTok{+}\StringTok{ }\KeywordTok{geom\_point}\NormalTok{(}\DataTypeTok{size=}\DecValTok{1}\NormalTok{) }\OperatorTok{+}
\StringTok{  }\KeywordTok{geom\_errorbar}\NormalTok{(}\KeywordTok{aes}\NormalTok{(}\DataTypeTok{x =}\NormalTok{ drug\_name, }\DataTypeTok{ymin =}\NormalTok{ ci95\_lower, }\DataTypeTok{ymax =}\NormalTok{ ci95\_upper), }\DataTypeTok{width=}\FloatTok{0.2}\NormalTok{) }\OperatorTok{+}
\StringTok{  }\KeywordTok{xlab}\NormalTok{(}\StringTok{""}\NormalTok{) }\OperatorTok{+}
\StringTok{  }\KeywordTok{ylab}\NormalTok{(}\StringTok{"Estimated effect"}\NormalTok{) }\OperatorTok{+}
\StringTok{  }\KeywordTok{coord\_flip}\NormalTok{()}
\end{Highlighting}
\end{Shaded}



\end{document}
